%!TEX TS-program = xelatex
%!TEX encoding = UTF-8 Unicode
% Awesome CV LaTeX Template
%
% This template has been downloaded from:
% https://github.com/posquit0/Awesome-CV
%
% Author:
% Claud D. Park <posquit0.bj@gmail.com>
% http://www.posquit0.com
%
% Template license:
% CC BY-SA 4.0 (https://creativecommons.org/licenses/by-sa/4.0/)
%


%%%%%%%%%%%%%%%%%%%%%%%%%%%%%%%%%%%%%%
%     Configuration
%%%%%%%%%%%%%%%%%%%%%%%%%%%%%%%%%%%%%%
%%% Themes: Awesome-CV
\documentclass[]{awesome-cv}
\usepackage{textcomp}
%%% Override a directory location for fonts(default: 'fonts/')
\fontdir[fonts/]

%%% Configure a directory location for sections
\newcommand*{\sectiondir}{resume/}

%%% Override color
% Awesome Colors: awesome-emerald, awesome-skyblue, awesome-red, awesome-pink, awesome-orange
%                 awesome-nephritis, awesome-concrete, awesome-darknight
%% Color for highlight
% Define your custom color if you don't like awesome colors
\colorlet{awesome}{awesome-red}
%\definecolor{awesome}{HTML}{CA63A8}
%% Colors for text
%\definecolor{darktext}{HTML}{414141}
%\definecolor{text}{HTML}{414141}
%\definecolor{graytext}{HTML}{414141}
%\definecolor{lighttext}{HTML}{414141}

%%% Override a separator for social informations in header(default: ' | ')
%\headersocialsep[\quad\textbar\quad]
    \begin{document}
    
%%%%%%%%%%%%%%%%%%%%%%%%%%%%%%%%%%%%%%
%     Profile
%%%%%%%%%%%%%%%%%%%%%%%%%%%%%%%%%%%%%%
\begin{center}
	\headerfirstnamestyle{Zeeve} \headerlastnamestyle{Rogoszinski} \\
	\vspace{2mm}
	{\faEnvelope\ zero@umd.edu} | {\faMapMarker\ College Park, MD 20742} | {\faLink\ https://www.astro.umd.edu/\textasciitilde{}zero/}
\end{center}
%%%%%%%%%%%%%%%%%%%%%%%%%%%%%%%%%%%%%%
%     Education
%%%%%%%%%%%%%%%%%%%%%%%%%%%%%%%%%%%%%%
\cvsection{Education}
\begin{cventries}
	\cventry
	{Ph.D. in Astronomy}
	{University of Maryland}
	{College Park, MD}
	{Aug 2020 (expected)}
	{}
	
	\vspace{-7mm}
	\cventry
	{M.S. in Astronomy}
	{}
	{}
	{Dec 2016}
	{}
	
	\vspace{-6mm}
	\cventry
	{B.A. in Astronomy \& Physics}
	{Vassar College}
	{Poughkeepsie, NY}
	{Jun 2014}
	{}
\end{cventries}

\vspace{-5mm}

\cvsection{Skills}
\begin{cventries}
	\vspace{-3mm}
	\cventry
	{}
	{\def\arraystretch{1.15}{\begin{tabular}{ l l }
		Programming Languages (proficient):  & {\skill{ Python, C, \LaTeX, Mathematica, shell scripting}} \\
		Programming Languages (novice):  & {\skill{ HTML/CSS}} \\
		Tools \& Software:  & {\skill{ Numpy, Matplotlib, Pandas, Scikit-learn, SciPy, Seaborn}} \\
		& {\skill{ Git, Jupyter Notebook, Microsoft Office, Slurm, Unix/Linux}} \\
		Spoken Languages:  & {\skill{ English (native), Hebrew (advanced)}}
		\end{tabular}}}
	{}
	{}
	{}
\end{cventries}

\vspace{-9mm}

\cvsection{Research Experience}
\begin{cventries}
		\cventry
		{Advisor: Dr. Douglas Hamilton}
		{Ann G. Wylie Dissertation Fellow/{\color{red}NASA Earth and Space Science Fellow}}
		{U Maryland}
		{present}
		{\vspace{-3mm}
			\begin{itemize}
				\item Responsible for model development, execution, and visualization of C based simulations for the evolution of planetary spin-states via spin-orbit resonances, gas accretion, and collisions. \vspace{0.7mm}
				\item Developed Python post-processing tools for data aggregation (up to 1-10 TB), visualization, and statistical analysis. \vspace{0.7mm}
				\item Repurposed an N-body simulator using a Python wrapper to calculate the evolution of satellite orbits after 100s of collisions. \vspace{0.7mm}
				\item Published a novel explanation for Uranus's and Neptune's tilts that both reduces the mass and number of subsequent impacts, and preserves the planets' spin periods. Reprints and additional information can be found on my website. \vspace{0.7mm}
			\end{itemize}
		}
	
	\vspace{-6mm}
	\cventry
	{Advisor: Dr. John Hewitt}
	{Summer Intern}
	{NASA GSFC}
	{2014}
	{\vspace{-3mm}
		\begin{itemize}
			\item Developed a Python image processing and analysis script to study cosmic ray origins in supernova remnants. 
		\end{itemize}
	}
	
	\vspace{-6mm}
	\cventry
	{Advisor: Dr. Debra Elmegreen}
	{Senior Thesis}
	{Vassar College}
	{2013-2014}
	{\vspace{-3mm}
		\begin{itemize}
			\item Analyzed elliptical galaxy data to find correlations between structure and star formation rates.
		\end{itemize}
	}
	
	\vspace{-6mm}
	\cventry
	{Advisor: Dr. Jay Pasachoff}
	{Keck Northeast Astronomy Consortium Summer Research Fellow}
	{Williams College}
	{2013}
	{\vspace{-3mm}
		\begin{itemize}
			\item Processed and analyzed raw images from the 2012 transit of Venus to explain the black-drop effect.
		\end{itemize}}
\end{cventries}

\vspace{-6mm}

\cvsection{Fellowships \& Awards}
\begin{cvhonors}
	\cvhonor
	{Ann G. Wylie Dissertation Fellowship}
	{}
	{U Maryland}
	{2020}
	\cvhonor
	{NASA Earth and Space Science Fellowship}
	{28 out of 180 selected}
	{NASA}
	{2016 - 2019}
	\cvhonor
	{Hartmann Student Travel Grant}
	{}
	{AAS}
	{2016}
	\cvhonor
	{Departmental Honors in Astronomy}
	{}
	{Vassar College}
	{2014}
	\cvhonor
	{Departmental Honors in Physics}
	{}
	{Vassar College}
	{2014}
	\cvhonor
	{General Honors}
	{}
	{Vassar College}
	{2014}
	\cvhonor
	{Sigma Xi}
	{}
	{}
	{2014}
	\cvhonor
	{Ethel Hickox Pollard Memorial Physics Award}
	{}
	{Vassar College}
	{2013}
	\cvhonor
	{Janet Murray \textquotesingle{}31 Memorial Scholarship}
	{}
	{Vassar College}
	{2013}
\end{cvhonors}

\vspace{-0mm}
\cvsection{Publications}
\begin{cventries}
	\cventry
	{Rogoszinski, Z., Hamilton D. P., 2020, ApJ. arXiv:1908.10969}
	{Tilting Ice Giants with a Spin-Orbit Resonance}
	{}
	{}
	{}
	
	\vspace{-5mm}
	\cventry
	{Rogoszinski, Z., Hamilton D. P., 2020, in preparation}
	{Tilting Uranus: Collisions vs. Spin-Orbit Resonance}
	{}
	{}
	{}
	
	\vspace{-6mm}
\end{cventries}

\cvsection{Presentations}
\begin{cventries}
	\cventry
	{Rogoszinski, Z., Hamilton D. P.}
	{Tilting Ice Giants with Circumplanetary Disks}
	{Division of Dynamical Astronomy}
	{Jun 2019}
	{}
	
	\vspace{-5mm}
	\cventry
	{Rogoszinski, Z., Hamilton D. P.}
	{Using collisions and resonances to tilting Uranus}
	{American Astronomical Society }
	{Jan 2018}
	{}
	
	\vspace{-5mm}
	\cventry
	{Rogoszinski, Z., Hamilton D. P.}
	{Continuing the investigation to tilting Uranus with a secular spin-orbit resonance}
	{Division of Planetary Science }
	{Oct 2017}
	{}
	
	\vspace{-5mm}
	\cventry
	{Rogoszinski, Z., Hamilton D. P.}
	{Tilting Uranus without a Collision}
	{AstroCon DC }
	{Jul 2017}
	{}	
	
	\vspace{-6mm}
\end{cventries}
\cvsection{Posters}
\begin{cventries}
	\cventry
	{Rogoszinski, Z., Hamilton D. P.}
	{Can The Spin Rates of Irregular Satellites Provide Constraints To Their Formation Histories? }
	{EPSC-DPS Joint Meeting}
	{Sept 2019}
	{}
	
	\vspace{-5mm}
	\cventry
	{Rogoszinski, Z., Hamilton D. P.}
	{How do collisions shape the orbits of irregular satellites?}
	{Division of Planetary Science}
	{Oct 2018}
	{}
	
	\vspace{-5mm}
	\cventry
	{Rogoszinski, Z., Hamilton D. P.}
	{Why is it so difficult to tilt Uranus?}
	{Division of Dynamical Astronomy}
	{Apr 2018}
	{}
	
	\vspace{-5mm}
	\cventry
	{Rogoszinski, Z., Hamilton D. P.}
	{Tilting Uranus without a Collision}
	{Division of Planetary Science}
	{Oct 2016}
	{}
	
	\vspace{-5mm}
	\cventry
	{Rogoszinski, Z., Hewitt, J. W.}
	{Constraining Cosmic Ray Origins Through Spectral Radio Breaks In Supernova Remnants }
	{American Astronomical Society}
	{Jan 2015}
	{}
	
	\vspace{-5mm}
	\cventry
	{Rogoszinski, Z., Pasachoff, J. M.}
	{Observations of the Black-Drop Effect at the 2012 Transit of Venus }
	{American Astronomical Society}
	{Jan 2014}
	{}
	
	\vspace{-5mm}
\end{cventries}

\cvsection{Services \& Internships}
\begin{cventries}
	\cventry
	{Volunteered with the GRAD-MAP program by assisting with outreach, and helping to plan the Winter Workshop. GRAD-MAP is a diversity initiative and graduate student led organization by the Astronomy and Physics departments dedicated to sustaining ties between UMD and other minority serving institutions. For more information, visit: www.umdgradmap.org}
	{GRAD-MAP Member}
	{U Maryland}
	{2015-2018}
	{}
	
	\vspace{-5mm}
	\cventry
	{A secretary position at a NASA peer review panel for annual proposals. These are usually reserved for early scientists to observe and learn from the proposal decision process.}
	{Executive Secretary}
	{NASA}
	{2017, 2018}
	{}
	
	\vspace{-5mm}
	\cventry
	{Maintained and operated the school's observatory.}
	{Observatory Assistant}
	{Vassar College}
	{2010-2012}
	{}
	
	\vspace{-5mm}
\end{cventries}

\cvsection{Teaching}
\begin{cventries}
	\cventry
	{Supervisors: Grace Deming, Dr. Douglas Hamilton, Dr. Lee Mundy, Dr. Eliza Kempton}
	{Astronomy 101 TA}
	{U Maryland}
	{2014-2016, Fall 2019}
	{}
	
	\vspace{-5mm}
	\cventry
	{Supervisor: Dr. Debra Elmegreen}
	{Academic Astronomy Intern}
	{Vassar College}
	{2013-2014}
	{}
	
	\vspace{-5mm}
	\cventry
	{Supervisor: Dr. Jay Pasachoff}
	{Teaching Assistant}
	{Williams College Planetarium}
	{Summer 2013}
	{}
	%\vspace{-10mm}
\end{cventries}
\end{document}