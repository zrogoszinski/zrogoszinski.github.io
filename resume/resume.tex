%!TEX TS-program = xelatex
%!TEX encoding = UTF-8 Unicode
% Awesome CV LaTeX Template
%
% This template has been downloaded from:
% https://github.com/posquit0/Awesome-CV
%
% Author:
% Claud D. Park <posquit0.bj@gmail.com>
% http://www.posquit0.com
%
% Template license:
% CC BY-SA 4.0 (https://creativecommons.org/licenses/by-sa/4.0/)
%


%%%%%%%%%%%%%%%%%%%%%%%%%%%%%%%%%%%%%%
%     Configuration
%%%%%%%%%%%%%%%%%%%%%%%%%%%%%%%%%%%%%%
%%% Themes: Awesome-CV
\documentclass[]{awesome-cv}
\usepackage{textcomp}
\usepackage{multicol}
\usepackage{enumitem}
\usepackage{amsfonts}

\newcommand*{\specialitem}{\item[$\checkmark$] \entrypositionstyle}

%%% Override a directory location for fonts(default: 'fonts/')
\fontdir[fonts/]

%%% Configure a directory location for sections
\newcommand*{\sectiondir}{resume/}

%%% Override color
% Awesome Colors: awesome-emerald, awesome-skyblue, awesome-red, awesome-pink, awesome-orange
%                 awesome-nephritis, awesome-concrete, awesome-darknight
%% Color for highlight
% Define your custom color if you don't like awesome colors
\colorlet{awesome}{awesome-red}
%\definecolor{awesome}{HTML}{CA63A8}
%% Colors for text
%\definecolor{darktext}{HTML}{414141}
%\definecolor{text}{HTML}{414141}
%\definecolor{graytext}{HTML}{414141}
%\definecolor{lighttext}{HTML}{414141}

%%% Override a separator for social informations in header(default: ' | ')
%\headersocialsep[\quad\textbar\quad]
    \begin{document}
    
%%%%%%%%%%%%%%%%%%%%%%%%%%%%%%%%%%%%%%
%     Profile
%%%%%%%%%%%%%%%%%%%%%%%%%%%%%%%%%%%%%%
\begin{center}
	\headerfirstnamestyle{Zeeve} \headerlastnamestyle{Rogoszinski} \\
	\vspace{2mm}
	{\faEnvelope\ zero@umd.edu} | {\faMapMarker\ College Park, MD 20742} | {\faLink\ https://www.astro.umd.edu/\textasciitilde{}zero/}
\end{center}
%%%%%%%%%%%%%%%%%%%%%%%%%%%%%%%%%%%%%%
%     Content
%%%%%%%%%%%%%%%%%%%%%%%%%%%%%%%%%%%%%%


\cvsection{Skills}
\begin{center}
	\begin{multicols}{3}
		\begin{itemize}
			\specialitem{Cluster/Super Computing}
			\specialitem{Reporting and Proposal Writing}
			\specialitem{Data Mining and Visualization}
		\end{itemize}
	\end{multicols}
\end{center}
\vspace{-7mm}
\begin{cventries}
	\cventry
	{}
	{\def\arraystretch{1.15}{\begin{tabular}{ l l }
		Programming Languages (proficient):  & {\skill{ Python, C, \LaTeX, Mathematica, shell scripting}} \\
		Programming Languages (novice):  & {\skill{ HTML/CSS}} \\
		Tools \& Software:  & {\skill{ Numpy, Matplotlib, Pandas, Scikit-learn, SciPy, Seaborn}} \\
							& {\skill{ Git, Jupyter Notebook, Microsoft Office, Slurm, Unix/Linux}} \\
		Spoken Languages:  & {\skill{ English (native), Hebrew (advanced)}}
		\end{tabular}}}
	{}
	{}
	{}
\end{cventries}

\vspace{-9mm}

\cvsection{Experience}
\begin{cventries}
	\cventry
	{Ann G. Wylie Dissertation Fellow/NASA Earth and Space Science Fellow/Graduate Student}
	{University of Maryland}
	{College Park, MD}
	{2014-present}
	{\vspace{-3mm}
		\begin{itemize}
			\item Responsible for model development, execution, and visualization of C based simulations for the evolution of planetary spin-states via spin-orbit resonances, gas accretion, and collisions. \vspace{0.7mm}
			\item Developed Python post-processing tools for data aggregation (up to 1-10 TB), visualization, and statistical analysis. \vspace{0.7mm}
			\item Repurposed an N-body simulator using a Python wrapper to calculate the evolution of satellite orbits after 100s of collisions. \vspace{0.7mm}
			\item Published a novel explanation for Uranus's and Neptune's tilts that both reduces the mass and number of subsequent impacts, and preserves the planets' spin periods. Reprints and additional information can be found on my website. \vspace{0.7mm}
			\item Presented my findings at several national and divisional meetings in the US and abroad. \vspace{0.7mm}
			\item Volunteered with the GRAD-MAP program by assisting with outreach, and helping to plan the Winter Workshop. GRAD-MAP is a diversity initiative and graduate student led organization by the Astronomy and Physics departments dedicated to sustaining ties between UMD and other minority serving institutions. For more information, visit: www.umdgradmap.org \vspace{0.7mm}
		\end{itemize}
	}

	\vspace{-6mm}
	\cventry
	{Summer Intern}
	{NASA Goddard Space Flight Center}
	{Greenbelt, MD}
	{2014}
	{\vspace{-3mm}
		\begin{itemize}
			\item Developed a Python image processing and analysis script to study cosmic ray origins in supernova remnants. 
	\end{itemize}
	}
	
	\vspace{-6mm}
	\cventry
	{Senior Thesis}
	{Vassar College}
	{Poughkeepsie, NY}
	{2013-2014}
	{\vspace{-3mm}
		\begin{itemize}
			\item Analyzed elliptical galaxy data to find correlations between structure and star formation rates.
	\end{itemize}
	}
	
	\vspace{-6mm}
	\cventry
	{Keck Northeast Astronomy Consortium Summer Research Fellow}
	{Williams College}
	{Williamstown, MA}
	{2013}
	{\vspace{-3mm}
		\begin{itemize}
			\item Processed and analyzed raw images from the 2012 transit of Venus to explain the black-drop effect.
	\end{itemize}}
\end{cventries}

\vspace{-7mm}

\cvsection{Education}
\begin{cventries}
	\cventry
	{Ph.D. in Astronomy}
	{University of Maryland}
	{College Park, MD}
	{Aug 2020 (expected)}
	{Advisor: Dr. Douglas Hamilton}
	
	\vspace{-2mm}
	\cventry
	{M.S. in Astronomy}
	{}
	{}
	{Dec 2016}
	{}
	
	\vspace{-5mm}
	\cventry
	{B.A. in Astronomy \& Physics (Graduated with Departmental and General Honors)}
	{Vassar College}
	{Poughkeepsie, NY}
	{Jun 2014}
	{Senior Thesis Advisor: Dr. Debra Elmegreen}
\end{cventries}

\vspace{-2mm}

\cvsection{Teaching}
\begin{cventries}

	\cventry
	{Supervisors: Grace Deming, Dr. Douglas Hamilton, Dr. Lee Mundy, Dr. Eliza Kempton}
	{Astronomy 101 TA}
	{U Maryland}
	{2014-2016, Fall 2019}
	{}
		
	\vspace{-6mm}
	\cventry
	{Supervisor: Dr. Debra Elmegreen}
	{Academic Astronomy Intern}
	{Vassar College}
	{2013-2014}
	{}
		
	\vspace{-6mm}
	\cventry
	{Supervisor: Dr. Jay Pasachoff}
	{Teaching Assistant}
	{Williams College Planetarium}
	{Summer 2013}
	{}
\end{cventries}



%\makecvfooter{}{}{}
\end{document}