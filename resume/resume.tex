%!TEX TS-program = xelatex
%!TEX encoding = UTF-8 Unicode
% Awesome CV LaTeX Template
%
% This template has been downloaded from:
% https://github.com/posquit0/Awesome-CV
%
% Author:
% Claud D. Park <posquit0.bj@gmail.com>
% http://www.posquit0.com
%
% Template license:
% CC BY-SA 4.0 (https://creativecommons.org/licenses/by-sa/4.0/)
%


%%%%%%%%%%%%%%%%%%%%%%%%%%%%%%%%%%%%%%
%     Configuration
%%%%%%%%%%%%%%%%%%%%%%%%%%%%%%%%%%%%%%
%%% Themes: Awesome-CV
\documentclass[]{awesome-cv}
\usepackage{textcomp}
\usepackage{multicol}
\usepackage{enumitem}
\usepackage{amsfonts}

\newcommand*{\specialitem}{\item[$\checkmark$] \entrypositionstyle}

%%% Override a directory location for fonts(default: 'fonts/')
\fontdir[fonts/]

%%% Configure a directory location for sections
\newcommand*{\sectiondir}{resume/}
\newcommand{\changeurlcolor}[1]{\hypersetup{urlcolor=#1}} 

%%% Override color
% Awesome Colors: awesome-emerald, awesome-skyblue, awesome-red, awesome-pink, awesome-orange
%                 awesome-nephritis, awesome-concrete, awesome-darknight
%% Color for highlight
% Define your custom color if you don't like awesome colors
\colorlet{awesome}{awesome-red}
%\definecolor{awesome}{HTML}{CA63A8}
%% Colors for text
%\definecolor{darktext}{HTML}{414141}
%\definecolor{text}{HTML}{414141}
%\definecolor{graytext}{HTML}{414141}
%\definecolor{lighttext}{HTML}{414141}

%%% Override a separator for social informations in header(default: ' | ')
%\headersocialsep[\quad\textbar\quad]
    \begin{document}
    
%%%%%%%%%%%%%%%%%%%%%%%%%%%%%%%%%%%%%%
%     Profile
%%%%%%%%%%%%%%%%%%%%%%%%%%%%%%%%%%%%%%
\begin{center}
	\headerfirstnamestyle{Zeeve} \headerlastnamestyle{Rogoszinski} \\
	\vspace{2mm}
	{\faEnvelope\ zero@umd.edu} | {\faMapMarker\ College Park, MD 20742} | {\faLink\ \changeurlcolor{black}\href{https://www.astro.umd.edu/~zero/}{https://www.astro.umd.edu/\textasciitilde{}zero/}}
\end{center}
%%%%%%%%%%%%%%%%%%%%%%%%%%%%%%%%%%%%%%
%     Content
%%%%%%%%%%%%%%%%%%%%%%%%%%%%%%%%%%%%%%


\cvsection{Skills}
\begin{center}
	\begin{multicols}{3}
		\begin{itemize}
			\specialitem{High Performance Computing}
			\specialitem{Reporting and Proposal Writing}
			\specialitem{Data Mining and Visualization}
		\end{itemize}
	\end{multicols}
\end{center}
\vspace{-7mm}
\begin{cventries}
	\cventry
	{}
	{\def\arraystretch{1.15}{\begin{tabular}{ l l }
		Programming Languages :  & {\skill{ Python, C, \LaTeX, Mathematica, shell scripting, HTML/CSS}} \\
		Tools \& Software:  & {\skill{ HDF5, Numpy, Matplotlib, Pandas, Scikit-learn, SciPy, Seaborn}} \\
							& {\skill{ Git, Jupyter Notebook, Microsoft Office, Slurm, Unix/Linux}} \\
		Spoken Languages:  & {\skill{ English (native), Hebrew (advanced)}}
		\end{tabular}}}
	{}
	{}
	{}
\end{cventries}

\vspace{-9mm}

\cvsection{Experience}
\begin{cventries}
	\cventry
	{University of Maryland, College Park, MD}
	{Doctoral Researcher and PhD Candidate}
	{Sept 2014 - present}
	{}
	{\vspace{-3mm}
		\begin{itemize}
			\item Explored additional explanations to the origins of planetary spin-states, with a focus on how Uranus was tilted on its side. \vspace{0.7mm}
			\item Developed C based simulations to model the evolution of tilts and spins of Uranus and Neptune via collisions and spin-orbit resonances. \vspace{0.7mm}
			\item Executed the DISCO moving-mesh magnetohydrodynamics software to model the spin evolution of gas giants via gas accretion. \vspace{0.7mm} 
			\item Developed Python post-processing tools for data aggregation (up to 1-10 TB), and visualization of these simulations. \vspace{0.7mm}
			\item Performed a rudementary statistical comparison of probable explanations for Uranus's and Neptune's spin-states. \vspace{0.7mm}
			\item Published a novel explanation for Uranus's and Neptune's tilts that both reduces the mass and number of subsequent impacts, and preserves the planets' spin periods. \vspace{0.7mm}
			\item Presented my findings at multiple conferences and meetings, and my work has been discussed in news articles such as Forbes and AAS Nova. \vspace{0.7mm}
			\item Repurposed an N-body simulator using a Python wrapper to calculate the evolution of satellite orbits after 100s of collisions. \vspace{0.7mm}
			\item Taught two lab sessions and one discussion section for the Astronomy 101 course over five semesters. \vspace{0.7mm}
		\end{itemize}
	}

	\vspace{-6mm}
	\cventry
	{NASA Goddard Space Flight Center, Greenbelt, MD}
	{Summer Researcher}
	{Jun 2014 - Aug 2014}
	{}
	{\vspace{-3mm}
		\begin{itemize}
			\item Interned with John Hewitt to study cosmic ray origins in supernova remnants. \vspace{0.7mm}
			\item Developed a Python image processing and analysis script to extract the total flux from three supernova remnants using Planck and WMAP data. \vspace{0.7mm}
			\item Compared the results to possible particle acceleration models to determine the process likely responsible for producing cosmic rays. \vspace{0.7mm}
			\item Presented a poster of my findings at the 225\textsuperscript{th} AAS meeting. 
		\end{itemize}
	}
	
	
	\vspace{-6mm}
	\cventry
	{Vassar College, Poughkeepsie, NY}
	{Senior Thesis Research}
	{Sept 2013 - May 2014}
	{}
	{\vspace{-3mm}
		\begin{itemize}
			\item Worked with Debra Elmegreen on an independent study of galaxy evolution using Hubble Deep Field optical images. \vspace{0.7mm}
			\item Analyzed the sizes and intensities of elliptical galaxies using IRAF to find correlations between structure and star formation rates. 
		\end{itemize}
	}
	
	\vspace{-6mm}
	\cventry
	{Williams College, Williamstown, MA}
	{Summer Research Fellow}
	{Jun 2013 - Aug 2013}
	{}
	{\vspace{-3mm}
		\begin{itemize}
			\item Worked with Jay Pasachoff as part of the Keck Northeast Astronomy Consortium to study the black-drop effect during the 2012 transit of Venus. \vspace{0.7mm}
			\item Processed and analyzed raw images using ImageJ and DS9 to measure the brightness of the planet during ingress. \vspace{0.7mm}
			\item Presented a poster of my findings at the 223\textsuperscript{rd} AAS meeting. 
	\end{itemize}}
\end{cventries}

\vspace{-7mm}

\cvsection{Education}
\begin{cventries}
	\cventry
	{\textbf{Ph.D. in Astronomy}, Advisor: Douglas Hamilton, Thesis: ``The Tilts and Spins of Planets and Moons''}
	{University of Maryland}
	{College Park, MD}
	{Aug 2020 (expected)}
	{}
	
	\vspace{-7mm}
	\cventry
	{\textbf{M.S. in Astronomy}, Advisor: Douglas Hamilton, Thesis: ``Tilting Uranus Without a Collision''}
	{}
	{}
	{Dec 2016}
	{}
	
	\vspace{-6mm}
	\cventry
	{\textbf{B.A. in Astronomy \& Physics}, Advisor: Debra Elmegreen, Thesis: ``Structure and Activity in	Hubble Deep Field Elliptical Galaxies''}
	{Vassar College}
	{Poughkeepsie, NY}
	{Jun 2014}
	{}
\end{cventries}

\vspace{-2mm}



%\makecvfooter{}{}{}
\end{document}