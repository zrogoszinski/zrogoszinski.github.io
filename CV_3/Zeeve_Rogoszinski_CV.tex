%!TEX TS-program = xelatex
%!TEX encoding = UTF-8 Unicode
% Awesome CV LaTeX Template
%
% This template has been downloaded from:
% https://github.com/posquit0/Awesome-CV
%
% Author:
% Claud D. Park <posquit0.bj@gmail.com>
% http://www.posquit0.com
%
% Template license:
% CC BY-SA 4.0 (https://creativecommons.org/licenses/by-sa/4.0/)
%


%%%%%%%%%%%%%%%%%%%%%%%%%%%%%%%%%%%%%%
%     Configuration
%%%%%%%%%%%%%%%%%%%%%%%%%%%%%%%%%%%%%%
%%% Themes: Awesome-CV
\documentclass[]{awesome-cv}
\usepackage{textcomp}
%%% Override a directory location for fonts(default: 'fonts/')
\fontdir[fonts/]

%%% Configure a directory location for sections
\newcommand*{\sectiondir}{resume/}
\newcommand{\changeurlcolor}[1]{\hypersetup{urlcolor=#1}} 

%%% Override color
% Awesome Colors: awesome-emerald, awesome-skyblue, awesome-red, awesome-pink, awesome-orange
%                 awesome-nephritis, awesome-concrete, awesome-darknight
%% Color for highlight
% Define your custom color if you don't like awesome colors
\colorlet{awesome}{awesome-red}
%\definecolor{awesome}{HTML}{CA63A8}
%% Colors for text
%\definecolor{darktext}{HTML}{414141}
%\definecolor{text}{HTML}{414141}
%\definecolor{graytext}{HTML}{414141}
%\definecolor{lighttext}{HTML}{414141}

%%% Override a separator for social informations in header(default: ' | ')
%\headersocialsep[\quad\textbar\quad]
    \begin{document}
    
%%%%%%%%%%%%%%%%%%%%%%%%%%%%%%%%%%%%%%
%     Profile
%%%%%%%%%%%%%%%%%%%%%%%%%%%%%%%%%%%%%%
\begin{center}
	\headerfirstnamestyle{Zeeve} \headerlastnamestyle{Rogoszinski} \\
	\vspace{2mm}
	{\faEnvelope\ zero@umd.edu} | {\faMapMarker\ College Park, MD 20742} | {\faLink\ \changeurlcolor{black}\href{https://zrogoszinski.github.io/}{https://zrogoszinski.github.io/}}
\end{center}
%%%%%%%%%%%%%%%%%%%%%%%%%%%%%%%%%%%%%%
%     Education
%%%%%%%%%%%%%%%%%%%%%%%%%%%%%%%%%%%%%%
\cvsection{Education}
\begin{cventries}
	\cventry
	{\textbf{Ph.D. in Astronomy}, Advisor: Douglas Hamilton, Thesis: ``The Tilts and Spins of Planets and Moons''}
	{University of Maryland}
	{College Park, MD}
	{Aug 2020}
	{}
	
	\vspace{-7mm}
	\cventry
	{\textbf{M.S. in Astronomy}, Advisor: Douglas Hamilton, Thesis: ``Tilting Uranus Without a Collision''}
	{}
	{}
	{Dec 2016}
	{}
	
	\vspace{-6mm}
	\cventry
	{\textbf{B.A. in Astronomy \& Physics}, Advisor: Debra Elmegreen, Thesis: ``Structure and Activity in	Hubble Deep Field Elliptical Galaxies''}
	{Vassar College}
	{Poughkeepsie, NY}
	{Jun 2014}
	{}
\end{cventries}

\vspace{-5mm}

\cvsection{Skills}
\begin{cventries}
	\vspace{-3mm}
	\cventry
	{}
	{\def\arraystretch{1.15}{\begin{tabular}{ l l }
		Analysis Skills: & {\skill {Data Mining, Data Visualization, High Performance Computing,}}\\
		& {\skill {Multiprocessing, Statistics and Probability}} \\
		Programming Languages :  & {\skill{ Python, C, \LaTeX, Mathematica, shell scripting, HTML/CSS}} \\
		Tools \& Software:  & {\skill{ HDF5, Numpy, Matplotlib, Pandas, Scikit-learn, SciPy, Seaborn}} \\
		& {\skill{ Git, Jupyter Notebook, Microsoft Office, Slurm, Unix/Linux}} \\
		Spoken Languages:  & {\skill{ English (native), Hebrew (advanced)}} \\
		%Additional Training: & {\skill{Self-Study on Machine Learning}}
		\end{tabular}}}
	{}
	{}
	{}
\end{cventries}

\vspace{-9mm}

\cvsection{Work Experience}
\begin{cventries}
	\cventry
	{Research Analyst}
	{Center for Naval Analyses}
	{Arlington, VA}
	{Oct 2020 - present}
	{}
\end{cventries}

\vspace{-5mm}

\cvsection{Research Experience}
\begin{cventries}
		\cventry
		{University of Maryland, College Park, MD}
		{Doctoral Researcher and PhD Candidate}
		{Sept 2014 - present}
		{}
		{\vspace{-3mm}
			\begin{itemize}
				\item Explored additional explanations to the origins of planetary spin-states, with a focus on how Uranus was tilted on its side. \vspace{0.7mm}
				\item Developed C based simulations to model the evolution of tilts and spins of Uranus and Neptune via collisions and spin-orbit resonances. \vspace{0.7mm}
				\item Executed the DISCO moving-mesh magnetohydrodynamics software to model the spin evolution of gas giants via gas accretion. \vspace{0.7mm} 
				\item Developed Python post-processing tools for large data set aggregation (up to 10 TB), and visualization of these simulations. \vspace{0.7mm}
				\item Performed a rudementary statistical comparison of probable explanations for Uranus's and Neptune's spin-states. \vspace{0.7mm}
				\item Published a novel explanation for Uranus's and Neptune's tilts that both reduces the mass and number of subsequent impacts, and preserves the planets' spin periods. \vspace{0.7mm}
				\item Presented my findings at multiple conferences and meetings both in the US and abroad. \vspace{0.7mm}
				\item Repurposed an N-body simulator using a Python wrapper to calculate the evolution of satellite orbits after 100s of collisions. \vspace{0.7mm}
			\end{itemize}
		}
	
	\vspace{-6mm}
	\cventry
	{NASA Goddard Space Flight Center, Greenbelt, MD}
	{Summer Researcher}
	{Jun 2014 - Aug 2014}
	{}
	{\vspace{-3mm}
		\begin{itemize}
			\item Interned with John Hewitt to study cosmic ray origins in supernova remnants. \vspace{0.7mm}
			\item Developed a Python image processing and analysis script to extract the total flux from three supernova remnants using Planck and WMAP data. \vspace{0.7mm}
			\item Compared the results to possible particle acceleration models to determine the process likely responsible for producing cosmic rays. \vspace{0.7mm}
			\item Presented a poster of my findings at the 225\textsuperscript{th} AAS meeting. 
		\end{itemize}
	}
	
	\vspace{-6mm}
	\cventry
	{Vassar College, Poughkeepsie, NY}
	{Senior Thesis Research}
	{Sept 2013 - May 2014}
	{}
	{\vspace{-3mm}
		\begin{itemize}
			\item Worked with Debra Elmegreen on an independent study of galaxy evolution using Hubble Deep Field optical images. \vspace{0.7mm}
			\item Analyzed the sizes and intensities of elliptical galaxies using IRAF to find correlations between structure and star formation rates. 
		\end{itemize}
	}
	
	\vspace{-6mm}
	\cventry
	{Williams College, Williamstown, MA}
	{Summer Research Fellow}
	{Jun 2013 - Aug 2013}
	{}
	{\vspace{-3mm}
		\begin{itemize}
			\item Worked with Jay Pasachoff as part of the Keck Northeast Astronomy Consortium to study the black-drop effect during the 2012 transit of Venus. \vspace{0.7mm}
			\item Processed and analyzed raw images using ImageJ and DS9 to measure the brightness of the planet during ingress. \vspace{0.7mm}
			\item Presented a poster of my findings at the 223\textsuperscript{rd} AAS meeting. 
		\end{itemize}}
\end{cventries}

\cvsection{Teaching and Leadership Experience}
\begin{cventries}
	\cventry
	{University of Maryland, College Park, MD}
	{Astronomy 101 TA}
	{Sept 2014 - May 2016, Fall 2019}
	{}
	{\vspace{-3mm}
		\begin{itemize}
		\item Taught two lab sessions and one discussion section for the Astronomy 101 course over five semesters. 
		\item Prepared lesson plans to compliment the week's topic for the discussion sections, including group activities and interactive demonstrations.
		\end{itemize}
	}
	
	\vspace{-6mm}
	\cventry
	{University of Maryland, College Park, MD}
	{GRAD-MAP Member}
	{Jan 2015 - Jan 2018}
	{}
	{\vspace{-3mm}
		\begin{itemize}
			\item Volunteered with the GRAD-MAP program by assisting with outreach, and helping to plan the Winter Workshop. 
			\item GRAD-MAP is a diversity initiative and graduate student led organization by the Astronomy and Physics departments dedicated to sustaining ties between UMD and other minority serving institutions.
		\end{itemize}
	}
	
	\vspace{-6mm}
	\cventry
	{NASA}
	{Executive Secretary}
	{2017, 2018}
	{}
	{\vspace{-3mm}
		\begin{itemize}
			\item A secretary position at NASA peer review panels for annual proposals. They are reserved for early scientists to observe and learn from the proposal decision process.
		\end{itemize}
	}
	
	\vspace{-6mm}
	\cventry
	{Williams College Planetarium, Williamstown, MA}
	{Planetarium Presenter}
	{Jun 2013 - Aug 2013}
	{}
	{\vspace{-3mm}
		\begin{itemize}
			\item Presented and operated the college's planetarium show for the public once a week.
		\end{itemize}
	}
	
	\vspace{-6mm}
	\cventry
	{Vassar College, Poughkeepsie, NY}
	{Observatory Assistant}
	{Sept 2010 - May 2012}
	{}
	{\vspace{-3mm}
		\begin{itemize}
			\item Maintained and operated the school's telescope multiple nights a week for student and professional research projects. 
		\end{itemize}
	}

	\vspace{-5mm}
\end{cventries}



\vspace{-2mm}

\cvsection{Fellowships \& Awards}
\begin{cvhonors}
	\cvhonor
	{Ann G. Wylie Dissertation Fellowship}
	{}
	{U Maryland}
	{2020}
	\cvhonor
	{NASA Earth and Space Science Fellowship}
	{28 out of 180 selected}
	{NASA}
	{2016 - 2019}
	\cvhonor
	{Hartmann Student Travel Grant}
	{}
	{AAS}
	{2016}
	\cvhonor
	{Departmental Honors in Astronomy}
	{}
	{Vassar College}
	{2014}
	\cvhonor
	{Departmental Honors in Physics}
	{}
	{Vassar College}
	{2014}
	\cvhonor
	{General Honors}
	{}
	{Vassar College}
	{2014}
	\cvhonor
	{Sigma Xi}
	{}
	{}
	{2014}
	\cvhonor
	{Ethel Hickox Pollard Memorial Physics Award}
	{}
	{Vassar College}
	{2013}
	\cvhonor
	{Janet Murray \textquotesingle{}31 Memorial Scholarship}
	{}
	{Vassar College}
	{2013}
\end{cvhonors}

\vspace{-2mm}
\cvsection{Publications}
\begin{cventries}
	\cventry
	{Rogoszinski, Z., 2020, PhD Thesis}
	{The Tilts and Spins of Planets and Moons}
	{}
	{}
	{}
	
	\vspace{-5mm}
	\cventry
	{Lawrence, S., Rogoszinski, Z., 2020, \href{https://arxiv.org/pdf/2004.14980.pdf}{arXiv:2004.14980}}
	{\href{https://ui.adsabs.harvard.edu/abs/2020arXiv200414980L/abstract}{The Brute-Force Search for Planet Nine}}
	{}
	{}
	{}
	
	\vspace{-5mm}
	\cventry
	{Rogoszinski, Z., Hamilton D. P., 2020, under review, \href{https://arxiv.org/pdf/2004.14913.pdf}{arXiv:2004.14913}}
	{\href{https://ui.adsabs.harvard.edu/abs/2020arXiv200414913R/abstract}{Tilting Uranus: Collisions vs. Spin-Orbit Resonance}}
	{}
	{}
	{}
	
	
	\vspace{-5mm}
	\cventry
	{Rogoszinski, Z., Hamilton D. P., 2020, ApJ. \href{https://arxiv.org/pdf/1908.10969.pdf}{arXiv:1908.10969}}
	{\href{https://ui.adsabs.harvard.edu/abs/2020ApJ...888...60R/abstract}{Tilting Ice Giants with a Spin-Orbit Resonance}}
	{}
	{}
	{}
	
	\vspace{-6mm}
\end{cventries}

\cvsection{Selected Posters and Presentations (4 out of 10)}
\begin{cventries}
	\cventry
	{Rogoszinski, Z., Hamilton D. P.}
	{Can The Spin Rates of Irregular Satellites Provide Constraints To Their Formation Histories? }
	{EPSC-DPS Joint Meeting}
	{Sept 2019}
	{}
	
	\vspace{-5mm}
	\cventry
	{Rogoszinski, Z., Hamilton D. P.}
	{{Tilting Ice Giants with Circumplanetary Disks}}
	{Division of Dynamical Astronomy}
	{Jun 2019}
	{}
	
%	\vspace{-5mm}
%	\cventry
%	{Rogoszinski, Z., Hamilton D. P.}
%	{Why is it so difficult to tilt Uranus?}
%	{Division of Dynamical Astronomy}
%	{Apr 2018}
%	{}
	
	\vspace{-5mm}
	\cventry
	{Rogoszinski, Z., Hamilton D. P.}
	{{Continuing the investigation to tilting Uranus with a secular spin-orbit resonance}}
	{Division of Planetary Science }
	{Oct 2017}
	{}	
	
	\vspace{-5mm}
	\cventry
	{Rogoszinski, Z., Hewitt, J. W.}
	{Constraining Cosmic Ray Origins Through Spectral Radio Breaks In Supernova Remnants }
	{American Astronomical Society}
	{Jan 2015}
	{}
	
	\vspace{-5mm}
\end{cventries}



%\cvsection{References}
%\begin{cventries}
%	\cventry
%	{Email: dphamil@umd.edu}
%	{Douglas Hamilton}
%	{Professor and Thesis Advisor}
%	{Phone: (301) 405-1548}
%	{}
%	
%	\vspace{-6mm}
%	\cventry
%	{Email: gsryan@umd.edu}
%	{Geoffrey Ryan}
%	{Post-Doc and Advisor}
%	{}
%	{}
%	%\vspace{-10mm}
%\end{cventries}
\end{document}