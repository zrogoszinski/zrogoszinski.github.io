%!TEX TS-program = xelatex
%!TEX encoding = UTF-8 Unicode
% Awesome CV LaTeX Template
%
% This template has been downloaded from:
% https://github.com/posquit0/Awesome-CV
%
% Author:
% Claud D. Park <posquit0.bj@gmail.com>
% http://www.posquit0.com
%
% Template license:
% CC BY-SA 4.0 (https://creativecommons.org/licenses/by-sa/4.0/)
%


%%%%%%%%%%%%%%%%%%%%%%%%%%%%%%%%%%%%%%
%     Configuration
%%%%%%%%%%%%%%%%%%%%%%%%%%%%%%%%%%%%%%
%%% Themes: Awesome-CV
\documentclass[]{awesome-cv}
\usepackage{textcomp}
%%% Override a directory location for fonts(default: 'fonts/')
\fontdir[fonts/]

%%% Configure a directory location for sections
\newcommand*{\sectiondir}{resume/}

%%% Override color
% Awesome Colors: awesome-emerald, awesome-skyblue, awesome-red, awesome-pink, awesome-orange
%                 awesome-nephritis, awesome-concrete, awesome-darknight
%% Color for highlight
% Define your custom color if you don't like awesome colors
\colorlet{awesome}{awesome-red}
%\definecolor{awesome}{HTML}{CA63A8}
%% Colors for text
%\definecolor{darktext}{HTML}{414141}
%\definecolor{text}{HTML}{414141}
%\definecolor{graytext}{HTML}{414141}
%\definecolor{lighttext}{HTML}{414141}

%%% Override a separator for social informations in header(default: ' | ')
%\headersocialsep[\quad\textbar\quad]
    \begin{document}
    
%%%%%%%%%%%%%%%%%%%%%%%%%%%%%%%%%%%%%%
%     Profile
%%%%%%%%%%%%%%%%%%%%%%%%%%%%%%%%%%%%%%
\begin{center}
	\headerfirstnamestyle{Zeeve} \headerlastnamestyle{Rogoszinski} \\
	\vspace{2mm}
	{\faEnvelope\ zero@astro.umd.edu} | {\faMapMarker\ College Park, MD 20742} | {\faLink\ https://www.astro.umd.edu/\textasciitilde{}zero/}
\end{center}
%%%%%%%%%%%%%%%%%%%%%%%%%%%%%%%%%%%%%%
%     Education
%%%%%%%%%%%%%%%%%%%%%%%%%%%%%%%%%%%%%%
\cvsection{Education}
\begin{cventries}
	\cventry
	{Ph.D. in Astronomy}
	{University of Maryland}
	{College Park, MD}
	{Aug 2020 (expected)}
	{Advisor: Dr. Douglas Hamilton}
	\cventry
	{M.S. in Astronomy}
	{University of Maryland}
	{College Park, MD}
	{Dec 2016}
	{}
	\cventry
	{B.A. in Astronomy \& Physics}
	{Vassar College}
	{Poughkeepsie, NY}
	{Jun 2010}
	{Senior Thesis Advisor: Dr. Debra Elmegreen}
\end{cventries}

\vspace{-2mm}

\cvsection{Skills}
\begin{cventries}
	\cventry
	{}
	{\def\arraystretch{1.15}{\begin{tabular}{ l l }
		Programming Languages:  & {\skill{ Python, C, Mathematica, \LaTeX, HTML}} \\
		Software:  & {\skill{ Unix, Jupyter Notebook, Microsoft Office}} \\
		\end{tabular}}}
	{}
	{}
	{}
\end{cventries}

\vspace{-7mm}
\cvsection{Fellowships \& Awards}
\begin{cvhonors}
	\cvhonor
	{Ann G. Wylie Dissertation Fellowship}
	{}
	{U Maryland}
	{2020}
	\cvhonor
	{NASA Earth and Space Science Fellowship}
	{28 out of 180 selected}
	{NASA}
	{2016 - 2019}
	\cvhonor
	{Hartmann Student Travel Grant}
	{}
	{AAS Division of Planetary Science}
	{2016}
	\cvhonor
	{Departmental Honors in Astronomy}
	{}
	{Vassar College}
	{2014}
	\cvhonor
	{Departmental Honors in Physics}
	{}
	{Vassar College}
	{2014}
	\cvhonor
	{General Honors}
	{}
	{Vassar College}
	{2014}
	\cvhonor
	{Sigma Xi}
	{}
	{}
	{2014}
	\cvhonor
	{Ethel Hickox Pollard Memorial Physics Award}
	{}
	{Vassar College}
	{2013}
	\cvhonor
	{Janet Murray \textquotesingle{}31 Memorial Scholarship}
	{}
	{Vassar College}
	{2013}
\end{cvhonors}
\cvsection{Publications}
\begin{cventries}
	\cventry
	{Rogoszinski, Z., Hamilton D. P. 2019, ApJ, in press, arXiv:1908.10969}
	{Tilting Ice Giants with a Spin-Orbit Resonance}
	{}
	{}
	{}
	
	\vspace{-5mm}
\end{cventries}

\cvsection{Presentations}
\begin{cventries}
	\cventry
	{Rogoszinski, Z., Hamilton D. P.}
	{Tilting Ice Giants with Circumplanetary Disks}
	{Division of Dynamical Astronomy}
	{Jun 2019}
	{}
	\cventry
	{Rogoszinski, Z., Hamilton D. P.}
	{Using collisions and resonances to tilting Uranus}
	{American Astronomical Society }
	{Jan 2018}
	{}
	\cventry
	{Rogoszinski, Z., Hamilton D. P.}
	{Continuing the investigation to tilting Uranus with a secular spin-orbit resonance}
	{Division of Planetary Science }
	{Oct 2017}
	{}
	\cventry
	{Rogoszinski, Z., Hamilton D. P.}
	{Tilting Uranus without a Collision}
	{AstroCon DC }
	{Jul 2017}
	{}	
	
	\vspace{-5mm}
\end{cventries}
\cvsection{Posters}
\begin{cventries}
	\cventry
	{Rogoszinski, Z., Hamilton D. P.}
	{How do collisions shape the orbits of irregular satellites?}
	{Division of Planetary Science}
	{Oct 2018}
	{}
	\cventry
	{Rogoszinski, Z., Hamilton D. P.}
	{Why is it so difficult to tilt Uranus?}
	{Division of Dynamical Astronomy}
	{Apr 2018}
	{}
	\cventry
	{Rogoszinski, Z., Hamilton D. P.}
	{Tilting Uranus without a Collision}
	{Division of Planetary Science}
	{Oct 2016}
	{}
	\cventry
	{Rogoszinski, Z., Hewitt, J. W.}
	{Constraining Cosmic Ray Origins Through Spectral Radio Breaks In Supernova Remnants }
	{American Astronomical Society}
	{Jan 2015}
	{NASA GSFC Summer Internship}
	\cventry
	{Rogoszinski, Z., Pasachoff, J. M.}
	{Observations of the Black-Drop Effect at the 2012 Transit of Venus }
	{American Astronomical Society}
	{Jan 2014}
	{Keck Northeast Astronomy Consortium Summer Research Fellow}
	
	\vspace{-2mm}
\end{cventries}
\cvsection{Teaching}
\begin{cventries}
	\cventry
	{Supervisor: Dr. Eliza Kempton}
	{Astronomy 101 TA}
	{U Maryland}
	{Fall 2019}
	{}
	\cventry
	{Supervisors: Grace Deming, Dr. Douglas Hamilton, Dr. Lee Mundy}
	{Astronomy 101 TA}
	{U Maryland}
	{2014-2016}
	{}
	\cventry
	{Supervisor: Dr. Debra Elmegreen}
	{Academic Astronomy Intern}
	{Vassar College}
	{2013-2014}
	{}
	\cventry
	{Supervisor: Dr. Jay Pasachoff}
	{Teaching Assistant}
	{Williams College Planetarium}
	{Summer 2013}
	{}
	
	\vspace{-5mm}
\end{cventries}
\cvsection{Activities}
\begin{cventries}
	\cventry
	{Volunteered with the GRAD-MAP program by assisting with outreach and helping to plan the Winter Workshop. I also maintained the website. For more information: https://www.umdgradmap.org/}
	{GRAD-MAP}
	{U Maryland}
	{2015-2018}
	{}
	\cventry
	{}
	{Executive Secretary}
	{NASA}
	{2017, 2018}
	{}
	\cventry
	{Maintained and operated the telescope}
	{Observatory Assistant}
	{Vassar College}
	{2010-2012}
	{}
	
	\vspace{-5mm}
\end{cventries}
\ 
\end{document}